%----------------------------------------------------------------------------------------
%	PACKAGES AND OTHER DOCUMENT CONFIGURATIONS
%----------------------------------------------------------------------------------------

\documentclass[letterpaper]{twentysecondcv} % a4paper for A4

%----------------------------------------------------------------------------------------
%	 PERSONAL INFORMATION
%----------------------------------------------------------------------------------------

% If you don't need one or more of the below, just remove the content leaving the command, e.g. \cvnumberphone{}

\profilepic{} % Profile picture, just write inside brackets the name of the file of the profile pic you want to use

\cvname{\huge Facundo Uriona\newline\newline} % Your name

\cvjobtitle{Full-Stack Developer} % Job title/career

\cvdate{24 June 1999} % Date of birth
\cvaddress{Argentina} % Short address/location, use \newline if more than 1 line is required
\cvsite{https://github.com/urionafacu} % Personal website
\cvmail{urionafacu@gmail.com} % Email address

%----------------------------------------------------------------------------------------

\begin{document}

%----------------------------------------------------------------------------------------
%	 ABOUT ME
%----------------------------------------------------------------------------------------
	
\aboutme{Versatile and adaptable Full-Stack Developer with experience in a fast-paced e-commerce startup environment. Demonstrated expertise in full-stack development, with a strong focus on mobile and web applications, backend systems, and cloud technologies. Proven ability to quickly learn and implement new technologies, lead projects from conception to deployment, and drive innovation in product development.} % To have no About Me section, just remove all the text and leave \aboutme{}

\interests{I want to work in a company which always faces new challenges, so that I will be able to grow professionally and personally. Besides, I would love to be part of a great team using new technologies to keep myself updated.}

\makeprofile % Print the sidebar

%----------------------------------------------------------------------------------------
%	 EXPERIENCE
%----------------------------------------------------------------------------------------

\section{Experience}
\newline\newline 
{\large\underline{Full Stack Developer: Vitau (YC)}}  \hspace*{105pt}  Oct-2020 - Present
\newline\newline
{\textit{Role: Product Software Developer}} \hspace*{131pt} Apr-2022 to Present
    \newline- Optimize responsive web UIs with React/Next.js, boosting UX.
    \newline- Develop and maintain backend systems using Django.
    \newline- Ensure reliability by implementing unit tests with pytest for each new feature and fix.
    \newline- Maintain and deploy React Native apps, enhancing stability and features.
    \newline- Deploy API Gateway and Lambda, streamlining serverless scalability.
    \newline- Set up secure SFTP on EC2 for data transfer and storage in S3.
    \newline- Containerize applications with Docker for AWS Fargate deployment.
    \newline- Develop a Go microservice for compressing multimedia files (PDF, PNG, JPG, etc), exploring new language capabilities.
    \newline- Schedule and manage frequent backend processes with Celery, improving efficiency and automation.
    \newline- Support non-technical teams by executing on-demand SQL queries in Metabase for business metrics analysis.
    \newline- Facilitate third-party integrations by providing technical support during API onboarding meetings.
    \newline- Provide ongoing support for product quality by addressing Sentry alerts in Slack error-reporting channels, actively resolving bugs across various platforms.
    \newline\newline\textit{\underline{Tools:} Django, Next.js, React Native, AWS, Postgres \underline{Methodology:} Agile}
    \newline\newline
{\textit{Role: Frontend \&\ Mobile Developer}} \hspace*{97pt} Oct-2020 to Apr-2022
    \newline- Developed mobile MVP in 6 months using Expo, React Native, and TypeScript.
    \newline- Applied Atomic Design for scalable UI; used Zustand/Immer for state management.
    \newline- Optimized data fetching with React Query; implemented reactive product search.
    \newline- Designed multi-step checkout flow (address, payment, confirmation).
    \newline- Set up Mixpanel tracking for user metrics and behavior analysis.
    \newline- Secured app with JWT authentication.
    \newline- Built HOCs and custom hooks for component reusability.
    \newline- Handled Expo pre-builds for native libraries; managed OTA and app store releases.
\newline\newline\textit{\underline{Tools:} React Native, TypeScript, Zustand, React Query \underline{Methodology:} Agile}
\newline\newline

%----------------------------------------------------------------------------------------
%	 EDUCATION
%----------------------------------------------------------------------------------------

\section{Education}

\begin{twenty} % Environment for a list with descriptions
	\twentyitem {}{Systems Engineering - Studies}{2017-2020}{Universidad Tecnológica Nacional (UTN)}

\end{twenty}


% \section{Courses}


%  \begin{twenty}
%  \twentyitem{}{First Certificate Exam}{2012}{University of Cambridge}
%   \twentyitem{}{Data Science - Level: Advanced}{2019}{Universidad Austral}
%  \end{twenty}

\end{document} 